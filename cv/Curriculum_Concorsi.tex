%%%%%%%%%%%%%%%%%%%%%%%%%%%%%%%%%%%%%%%%%
% Medium Length Professional CV
% LaTeX Template
% Version 2.0 (8/5/13)
%
% This template has been downloaded from:
% http://www.LaTeXTemplates.com
%
% Original author:
% Trey Hunner (http://www.treyhunner.com/)
%
% Important note:
% This template requires the resume.cls file to be in the same directory as the
% .tex file. The resume.cls file provides the resume style used for structuring the
% document.
%
%%%%%%%%%%%%%%%%%%%%%%%%%%%%%%%%%%%%%%%%%

%----------------------------------------------------------------------------------------
%	PACKAGES AND OTHER DOCUMENT CONFIGURATIONS
%----------------------------------------------------------------------------------------

\documentclass{resume} % Use the custom resume.cls style
\usepackage[T1]{fontenc}
\usepackage[utf8]{inputenc}
\usepackage[italian]{babel}
\usepackage{comment}
\usepackage{enumitem}
\setlist{nosep}
\usepackage[left=0.75in,top=0.6in,right=0.75in,bottom=0.6in]{geometry} % Document margins
\usepackage{natbib, bibentry}
\usepackage{tabularx}
\bibliographystyle{plain}
\name{Matteo Cardellini} % Your name
\address{Università degli Studi di Genova, Italia} % Your address
\address{matteo.cardellini@edu.unige.it \\ me@matteocardellini.it } % Your phone number and email


\nobibliography*
\begin{document}


%----------------------------------------------------------------------------------------
%	PUBBLICATIONS AND THESIS
%----------------------------------------------------------------------------------------

\begin{rSection}{Percorso di studi, professionale e accademico}

\begin{tabularx}{0.95\textwidth} {lp{14cm}}
 \textbf{2024-2025} & Assegnista di Ricerca presso il DIBRIS, Università degli Studi di Genova, Borsa di un anno dal titolo "Intelligenza Artificiale per il Dispacciamento Ferroviario".  \\
 \textbf{2023}  & Ricercatore in visita da Febbraio a Maggio 2023 all'University of Huddersfield, UK.\\
  \textbf{2022}  & Menzione Speciale del premio AIxIA Leonardo Lesmo per la migliore tesi magistrale italiana in Intelligenza Artificiale\\
 \textbf{2021-2024}  & Dottorato di Ricerca (conseguito con Lode) nel Dottorato Nazionale in Intelligenza Artificiale (Sede Amministrativa Politecnico di Torino, Sede Lavorativa Università di Genova) con borsa dal titolo "Pianificazione e schedulazione basate su metodologie di intelligenza artificiale in ambito ferroviario".\\
 \textbf{2020} & Borsa post-laurea di 5k euro dal titolo "Induzione e deduzione per la pianificazione del traffico ferroviario nelle stazioni di piccole e medie dimensioni", Università degli studi di Genova. \\
 \textbf{2019-2021}  & Laurea Magistrale (110/110 con Lode e Dignità di Stampa) in Computer Engineering (LM-32), curriculum in Artificial Intelligence and Human-Centered Computing, presso l'Università degli Studi di Genova.  \\
 \textbf{2016-2019}  & Laurea Triennale (110/110) in Ingegneria Informatica (L-8), presso l'Università degli Studi di Genova. \\
 \textbf{2019}  & Diploma di Maturità Scientifica - Liceo Scientifico Convitto C. Colombo, Genova. 
\end{tabularx}
\end{rSection}
\begin{rSection}{Dottorato di ricerca}
	Ho conseguito il Dottorato di Ricerca presso il Dottorato Nazionale in Intelligenza Artificiale (Sede Amministrativa Politecnico di Torino, Sede Lavorativa Università di Genova) con borsa dal titolo "Pianificazione e schedulazione basate su metodologie di intelligenza artificiale in ambito ferroviario". Ho conseguito con lode il titolo il 9 Gennaio 2025 con la tesi "Symbolic Pattern Planning".
\end{rSection}
\begin{rSection}{Incarichi  di insegnamento}
	\begin{tabularx}{0.95\textwidth} {lp{14cm}}
 \textbf{2025} & AI-Based Planning. 12h. 3CFU. PhD in Security, Risk and Vulnerability. Università degli Studi di Genova.  \\
 \textbf{2025} & Artificial Intelligence for Robotics II. 20h. 5CFU. ING-INF/05. Laurea Magistrale in Robotics Engineering (LM-32). Università degli Studi di Genova.  \\
  \textbf{2025} & Artificial Intelligence. 8h. 1CFU. ING-INF/05. Laurea Magistrale in Computer Engineering (LM-32). Università degli Studi di Genova.  \\
  \textbf{2025} & Fondamenti di Informatica. 70h. 7CFU. ING-INF/05. Laurea Triennale in Ingegneria Elettrica (L-9) e Ingegneria Chimica e di Processo (L-9). Università degli Studi di Genova.  \\
\end{tabularx}
\end{rSection}
\begin{rSection}{Tutoraggio}
	\begin{tabularx}{0.95\textwidth} {lp{14cm}}
 \textbf{2024} & Fondamenti di Informatica. 30h. ING-INF/05. Ingegneria Nautica.  \\
 \textbf{2023} & Fondamenti di Informatica. 30h. ING-INF/05. Ingegneria Nautica.  \\
 \textbf{2022} & Basi di Dati. 20h. ING-INF/05. Ingegneria Informatica \end{tabularx}
\end{rSection}
\pagebreak
\begin{rSection}{Relatore o correlatore di tesi}
	\begin{tabularx}{0.95\textwidth} {lp{14cm}}
 \textbf{2022} & A. Formica. Laurea Magistrale in Computer Engineering. "In-Station Train Dispatching via Artificial Intelligence Techniques: Optimisation, Rescheduling and Visualisation". Correlatore. Università di Genova.  \\
 \textbf{2022} & C. Ansaldo e N. Chiesa. Laurea Triennale in Ingegneria Informatica. "Artificial Intelligence Techniques for Solving the Shift Scheduling Problem". Correlatore. Università di Genova.
 \end{tabularx}
\end{rSection}

\begin{rSection}{Assegni di ricerca post-doc}
	\begin{tabularx}{0.95\textwidth} {lp{14cm}}
 \textbf{2024} & Assegno di ricerca post-doc di durata annuale dal titolo "Intelligenza Artificiale per il Dispacciamento Ferroviario". SSD ING-INF/05 Dal 04/11/2024 al 3/11/2025
 \end{tabularx}
\end{rSection}

\begin{rSection}{Partecipazione progetti di ricerca di enti privati}
	\begin{tabularx}{0.95\textwidth} {lp{14cm}}
 	\textbf{2024-2025} & Progetto dal titolo "Intelligenza Artificiale per il Dispacciamento Ferroviario" all'interno del RaidLab, laboratorio tra Hitachi Rail e Università di Genova. \\
  	\textbf{2021-2024} & Progetto dal titolo "Pianificazione e schedulazione basate su metodologie di intelligenza artificiale in ambito ferroviario" all'interno del RaidLab, laboratorio tra Hitachi Rail e Università di Genova. \\
   	\textbf{2019-2021} & Progetto dal titolo "Tecniche di intelligenza artificiale per il problema del dispacciamento dei treni in stazione" all'interno del RaidLab, laboratorio tra Hitachi Rail e Università di Genova.\\
    \textbf{2016-2019} & Progetto dal titolo "Induzione e deduzione per la pianificazione del traffico ferroviario nelle stazioni di piccole e medie dimensioni" all'interno del RaidLab, laboratorio tra Hitachi Rail e Università di Genova.\\
 \end{tabularx}
\end{rSection}

\begin{rSection}{Partecipazione progetti di ricerca internazionali}
	\begin{tabularx}{0.95\textwidth} {lp{14cm}}
 \textbf{2022-2025} & 2022-2025	EU Horizon Europe - ELSA - European Lighthouse on Secure and Safe	AI (150k Euro)
 \end{tabularx}
\end{rSection}

\begin{rSection}{Partecipazione progetti di ricerca nazionali}
	\begin{tabularx}{0.95\textwidth} {lp{14cm}}
 \textbf{?-?} & SERICS\\
  \textbf{?-?} & FAIR\\
 \end{tabularx}
\end{rSection}

\begin{rSection}{Relatore a conferenze internazionali}
	\begin{tabularx}{0.95\textwidth} {lp{14cm}}
 \textbf{2025-02-28} & Association for the Advancement of Artificial Intellgence (AAAI). Philadelphia, Pennsylvania, USA.\\
 \textbf{2024-10-15} & International Conference on Logic Programming (ICLP). Dallas, Texas, USA.\\
 \textbf{2024-02-24} & Association for the Advancement of Artificial Intelligence (AAAI). Vancouver, Canada.\\
 \textbf{2022-08-02} & Doctoral Consortium of the International Conference of Logic Programming (ICLP). Haifa, Israele.\\
 \textbf{2021-08-11} & International Conference of Automated Planning and Reasoning (ICAPS). Virtual.\\
 \end{tabularx}
 \end{rSection}
 

 \begin{rSection}{Relatore a conferenze nazionali}
	\begin{tabularx}{0.95\textwidth} {lp{14cm}}
 \textbf{2021-09-08} & Conferenza Italiana di Logica Computazionale (CILC).  Parma, Italy.\\
 \textbf{2022-10-28} & Doctoral Consortium alla Conferenza della Associazione Italiana per l'Intelligenza Artificiale (AIxIA). Udine, Italy.\\
 \textbf{2023-11-07} & Workshop Italiano di Planning e Scheduling (IPS) alla Conferenza della Associazione Italiana per l'Intelligenza Artificiale (AIxIA). Roma, Italy.
  \end{tabularx}
 \end{rSection}
 
 \begin{rSection}{Premi e riconoscimenti}
 \begin{tabularx}{0.95\textwidth} {lp{14cm}}

\textbf{2022}  & Menzione Speciale del premio Leonardo Lesmo per la migliore tesi magistrale italiana in Intelligenza Artificiale conferito dalla Associazione Italiana per l'Intelligenza Artificiale. 
  \end{tabularx}
\end{rSection}


 
  \begin{rSection}{Membro di comitati di programma}
	\begin{tabularx}{0.95\textwidth} {lp{14cm}}
 \textbf{2025} & Reviewer. European Conference of Artificial Intelligence \\
 \textbf{2025} & Reviewer. Journal of Applied Logic \\
 \textbf{2023-2025} & Reviewer. International Conference on Automated Planning and Scheduling \\
 \textbf{2023-2025} & Reviewer. International AAAI Conference on Artificial Intelligence 
  \end{tabularx}
 \end{rSection}
 
  \begin{rSection}{Organizzazione conferenze}
	\begin{tabularx}{0.95\textwidth} {lp{14cm}}
 \textbf{2022} & Organizzazione Locale della International Conference on Logic Programming and Non-monotonic Reasoning. Genova, Italy.
  \end{tabularx}
 \end{rSection}


 \begin{rSection}{Seminari e relazioni su invito}
 	\begin{tabularx}{0.95\textwidth} {lp{14cm}}
 \textbf{2025-01-25} & Università degli Studi di Genova. Seminario su "Symbolic Pattern Planning".\\
 \textbf{2024-07-16} & University of Oxford. Seminario su "Symbolic Pattern Planning".\\
 \textbf{2024-02-16} & Fondazione Bruno Kessler. Seminario su "Symbolic Pattern Planning".\\
 \textbf{2023-03-01} & University of Huddersfield. Seminario su "An ASP Framework for Efficient Urban Traffic Optimization".\\
 \end{tabularx}

\end{rSection}

\begin{rSection}{Articoli su riviste internazionali}
\begin{enumerate}[leftmargin=5mm]
	\item[J3] \textbf{Optimising Dynamic Traffic Distribution for Urban Networks with Answer Set Programming}. M. Cardellini, C. Dodaro, M. Maratea and M. Vallati - Theory and Practice of Logic Programming,  Volume 24, Issue 4, July 2024, pp. 825-843 - Scimago: Q2 on Artificial Intelligence
	\item[J2] \textbf{Solving Rehabilitation Scheduling Problems via a Two-Phase ASP approach}. M. Cardellini, P. De Nardi, C. Dodaro, G. Galat\`a, A. Giardini, M. Maratea, I. Porro. Theory and Practice of Logic Programming, Volume 24, Issue 2, March 2024, pp. 344-367 - Scimago: Q2 on Artificial Intelligence
	\item[J1] \textbf{Rescheduling Rehabilitation Sessions with Answer Set Programming}. M. Cardellini, C. Dodaro, G. Galat\`a, A. Giardini, M. Maratea, N. Nisopoli and I. Porro. Journal of Logic and Computation, Volume 33, Issue 3, April 2023, pp. 837-863 - Scimago: Q2 on Logic
\end{enumerate}
\end{rSection}
 
\begin{rSection}{contributi a convegni internazionali}
\begin{enumerate}[leftmargin=5mm]
	\item[C9] \textbf{Initial Condition Retrieving for Hybrid and Numeric Planning Problems}. M. Cardellini, M. Maratea, F. Percassi and M. Vallati. Proceedings of the 35th International Conference on Automated Planning and Scheduling. AAAI Press, 2025 - GGS: A, CORE: A$^{++}$
	
	
	\item[C8] \textbf{Temporal Numeric Planning with Patterns}. M. Cardellini and E. Giunchiglia. Proceedings of the 39th Annual AAAI Conference on Artificial Intelligence. AAAI Press, 2025 - GGS: A$^{++}$, CORE: A$^{++}$
	
	\item[C7] \textbf{Taming Discretised PDDL+ through Multiple Discretisations}. M. Cardellini, M. Maratea, F. Percassi, E. Scala and M. Vallati. Proceedings of the 34th International Conference on Automated Planning and Scheduling. AAAI Press, 2024 - GGS: A, CORE: A$^{++}$
	
	\item[C6] \textbf{Symbolic Numeric Planning With Patterns}. M. Cardellini, E. Giunchiglia and M. Maratea. Proceedings of the 38th Annual AAAI Conference on Artificial Intelligence. AAAI Press, 2024 - GGS: A$^{++}$, CORE: A$^{++}$
	
	\item[C5] \textbf{A Framework for Risk-Aware Routing of Connected Autonomous Vehicles via Artificial Intelligence}. M. Cardellini, C. Dodaro, M. Maratea, and M. Vallati. In Proceedings of the 26th IEEE International Conference on Intelligent Transportation Systems. IEEE, 2023 - MA: A$^-$
	
	%\item[C4] \textbf{A Two-Phase ASP Encoding for Solving Rehabilitation Scheduling}. M. Cardellini, P. De Nardi, C. Dodaro, G. Galat\'a, A. Giardini, M. Maratea and I. Porro. 2021. In Proceedings of the 5th International Joint Conference, RuleML+RR. Springer, 2021 - CORE: B
	
	
	\item[C3] \textbf{In-Station Train Movements Prediction: from Shallow to Deep Multi Scale Models}. G. Boleto, L. Oneto, M. Cardellini, M. Maratea, M. Vallati, R. Canepa, D. Anguita. In Proceedings of the 29th European Symposium on Artificial Neural Networks. i6doc, 2021 - GGS: B, CORE: B
	
	\item[C2] \textbf{An Efficient Hybrid Planning Framework for In-Station Train Dispatching}. M. Cardellini, M. Maratea, M. Vallati, G. Boleto, and L. Oneto. Proceedings of the 21st International Conference on Computational Science. Springer, 2021 - GGS: B, CORE: A
	
	\item[C1] \textbf{In-Station Train Dispatching: A PDDL+ Planning Approach}. M. Cardellini, M. Maratea, M. Vallati, G. Boleto, and L. Oneto. Proceedings of the 31st International Conference on Automated Planning and Scheduling. AAAI Press, 2021 - GGS: A, CORE: A$^{++}$
\end{enumerate}
\end{rSection}

 \begin{rSection}{tesi di dottorato}
	\begin{tabularx}{0.95\textwidth} {lp{14cm}}
 \textbf{2025} & Tesi di dottorato dal titolo "Symbolic Pattern Planning". Discussa il 9 Gennaio 2025.
  \end{tabularx}
 \end{rSection}

\end{document}
