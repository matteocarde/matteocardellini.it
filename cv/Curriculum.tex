%%%%%%%%%%%%%%%%%%%%%%%%%%%%%%%%%%%%%%%%%
% Medium Length Professional CV
% LaTeX Template
% Version 2.0 (8/5/13)
%
% This template has been downloaded from:
% http://www.LaTeXTemplates.com
%
% Original author:
% Trey Hunner (http://www.treyhunner.com/)
%
% Important note:
% This template requires the resume.cls file to be in the same directory as the
% .tex file. The resume.cls file provides the resume style used for structuring the
% document.
%
%%%%%%%%%%%%%%%%%%%%%%%%%%%%%%%%%%%%%%%%%

%----------------------------------------------------------------------------------------
%	PACKAGES AND OTHER DOCUMENT CONFIGURATIONS
%----------------------------------------------------------------------------------------

\documentclass{resume} % Use the custom resume.cls style
\usepackage[T1]{fontenc}
\usepackage[utf8]{inputenc}
\usepackage[italian]{babel}
\usepackage{comment}
\usepackage{enumitem}
\setlist{nosep}
\usepackage[left=0.75in,top=0.6in,right=0.75in,bottom=0.6in]{geometry} % Document margins
\usepackage{natbib, bibentry}
\usepackage{tabularx}
\bibliographystyle{plain}
\name{Matteo Cardellini} % Your name
\address{Università degli Studi di Genova, Italia} % Your address
\address{matteo.cardellini@unige.it \\ me@matteocardellini.it } % Your phone number and email


\nobibliography*
\begin{document}


%----------------------------------------------------------------------------------------
%	PUBBLICATIONS AND THESIS
%----------------------------------------------------------------------------------------

\begin{rSection}{Educational, Professional and Academic Background}

\begin{tabularx}{0.95\textwidth} {lp{14cm}}
	\textbf{2025} & Honourable Mention for an Outstanding Dissertation Award at ICAPS 2025.\\
 \textbf{2025} & Appointed Scientific Coordinator of the Joint DiDiLab Laboratory between the University of Genoa and ParvaSoft S.p.a.  \\
 \textbf{2025-2028} & Researcher at DIBRIS, University of Genoa.  \\
 \textbf{2024-2025} & Postdoctoral Research Fellow at DIBRIS, University of Genoa, one-year grant titled "Artificial Intelligence for Railway Dispatching".  \\
 \textbf{2023}  & Visiting Researcher from February to May 2023 at the University of Huddersfield, UK.\\
  \textbf{2022}  & Special Mention of the AIxIA Leonardo Lesmo Award for the best Italian Master's thesis in Artificial Intelligence\\
 \textbf{2021-2024}  & PhD (awarded with honors) in the National PhD in Artificial Intelligence (Administrative site: Polytechnic University of Turin, Working site: University of Genoa) with a grant titled "Planning and Scheduling Based on Artificial Intelligence Methodologies in the Railway Domain".\\
 \textbf{2020} & Post-graduate fellowship of EUR5k titled "Induction and Deduction for Railway Traffic Planning in Small and Medium-Sized Stations", University of Genoa. \\
 \textbf{2019-2021}  & Master’s Degree (110/110 with Honors and Publication Dignity) in Computer Engineering (LM-32), curriculum in Artificial Intelligence and Human-Centered Computing, University of Genoa.  \\
 \textbf{2016-2019}  & Bachelor’s Degree (110/110) in Computer Engineering (L-8), University of Genoa. \\
 \textbf{2016}  & High School Diploma in Scientific Studies – Liceo Scientifico Convitto C. Colombo, Genoa. 
\end{tabularx}
\end{rSection}
\begin{rSection}{PhD}
	I obtained my PhD in the National PhD in Artificial Intelligence (Administrative site: Polytechnic University of Turin, Working site: University of Genoa) with a grant titled "Planning and Scheduling Based on Artificial Intelligence Methodologies in the Railway Domain". I was awarded the title with honors on January 9, 2025, with the thesis "Symbolic Pattern Planning".
\end{rSection}
\begin{rSection}{Teaching}
	\begin{tabularx}{0.95\textwidth} {lp{14cm}}
 \textbf{2025} & Algorithms. 48h. 6CFU. Bachelor's Degree in Computer Engineering (L-8). University of Genoa.  \\
 \textbf{2025} & AI-Based Planning. 12h. 3CFU. PhD in Security, Risk and Vulnerability. University of Genoa.  \\
 \textbf{2025} & Artificial Intelligence for Robotics II. 20h. 5CFU. ING-INF/05. Master’s Degree in Robotics Engineering (LM-32). University of Genoa.  \\
  \textbf{2025} & Artificial Intelligence. 8h. 1CFU. ING-INF/05. Master’s Degree in Computer Engineering (LM-32). University of Genoa.  \\
  \textbf{2025} & Fundamentals of Computer Science. 70h. 7CFU. ING-INF/05. Bachelor’s Degree in Electrical Engineering (L-9) and Chemical and Process Engineering (L-9). University of Genoa.  \\
\end{tabularx}
\pagebreak
\end{rSection}
\begin{rSection}{Tutoring}
	\begin{tabularx}{0.95\textwidth} {lp{14cm}}
 \textbf{2024} & Fundamentals of Computer Science. 30h. ING-INF/05. Naval Engineering.  \\
 \textbf{2023} & Fundamentals of Computer Science. 30h. ING-INF/05. Naval Engineering.  \\
 \textbf{2022} & Databases. 20h. ING-INF/05. Computer Engineering \end{tabularx}
\end{rSection}
\begin{rSection}{Thesis Supervisor or Co-Supervisor}
	\begin{tabularx}{0.95\textwidth} {lp{14cm}}
 \textbf{2022} & A. Formica. Master’s Degree in Computer Engineering. "In-Station Train Dispatching via Artificial Intelligence Techniques: Optimisation, Rescheduling and Visualisation". Co-supervisor. University of Genoa.  \\
 \textbf{2022} & C. Ansaldo and N. Chiesa. Bachelor’s Degree in Computer Engineering. "Artificial Intelligence Techniques for Solving the Shift Scheduling Problem". Co-supervisor. University of Genoa.
 \end{tabularx}
\end{rSection}

\begin{rSection}{Post-Doc Research Grants}
	\begin{tabularx}{0.95\textwidth} {lp{14cm}}
 \textbf{2024} & One-year post-doc research grant titled "Artificial Intelligence for Railway Dispatching". SSD ING-INF/05 From 04/11/2024 to 03/11/2025
 \end{tabularx}
\end{rSection}

\begin{rSection}{Periods at Foreign Universities and Research Centers}
	\begin{tabularx}{0.95\textwidth} {lp{14cm}}
 \textbf{2023} & University of Huddersfield. Visiting Researcher. From 01/02/2023 to 15/05/2023. Worked in Prof. Mauro Vallati’s group at the Centre for Planning, Autonomy and Representation of Knowledge (PARK).
 \end{tabularx}
\end{rSection}

\begin{rSection}{Participation in Research Projects with Private Entities}
	\begin{tabularx}{0.95\textwidth} {lp{14cm}}
 	\textbf{2025-2028} & Scientific Coordinator of the Joint DiDiLab Laboratory between the University of Genoa and ParvaSoft S.p.a. aimed at research and training on artificial intelligence applied to integrated logistics management. \\
 	\textbf{2024-2025} & Project titled "Artificial Intelligence for Railway Dispatching" within the RaidLab, a joint laboratory between Hitachi Rail and the University of Genoa. \\
  	\textbf{2021-2024} & Project titled "Planning and Scheduling Based on Artificial Intelligence Methodologies in the Railway Domain" within the RaidLab, a joint laboratory between Hitachi Rail and the University of Genoa. \\
   	\textbf{2019-2021} & Project titled "Artificial Intelligence Techniques for the Train Dispatching Problem in Stations" within the RaidLab, a joint laboratory between Hitachi Rail and the University of Genoa.\\
    \textbf{2016-2019} & Project titled "Induction and Deduction for Railway Traffic Planning in Small and Medium-Sized Stations" within the RaidLab, a joint laboratory between Hitachi Rail and the University of Genoa.\\
 \end{tabularx}
\end{rSection}

\begin{rSection}{Participation in International Research Projects}
I carried out research activities within the following international projects:\\
	\begin{tabularx}{0.95\textwidth} {lp{14cm}}
 \textbf{2022-2025} & EU Horizon Europe. European Lighthouse on Secure and Safe AI (ELSA)
 \end{tabularx}
\end{rSection}

\begin{rSection}{Participation in National Research Projects}
I carried out research activities within the following national projects:\\
	\begin{tabularx}{0.95\textwidth} {lp{14cm}}
  \textbf{2024-2024} & Extended Partnership. Future Artificial Intelligence Research (FAIR)\\
 \textbf{2023-2028} & Extended Partnership. SEcurity and RIghts In the CyberSpace (SERICS)\\
 \end{tabularx}
\end{rSection}

\break

\begin{rSection}{Talks at International Conferences}
	\begin{tabularx}{0.95\textwidth} {lp{14cm}}
 \textbf{2025-08-20} & International Joint Conference on Artificial Intelligence (IJCAI). Montreal, Quebec, Canada.\\
 \textbf{2025-02-28} & Association for the Advancement of Artificial Intelligence (AAAI). Philadelphia, Pennsylvania, USA.\\
 \textbf{2024-10-15} & International Conference on Logic Programming (ICLP). Dallas, Texas, USA.\\
 \textbf{2024-02-24} & Association for the Advancement of Artificial Intelligence (AAAI). Vancouver, British Columbia, Canada.\\
 \textbf{2022-08-02} & Doctoral Consortium of the International Conference on Logic Programming (ICLP). Haifa, Israel.\\
 \textbf{2021-08-11} & International Conference on Automated Planning and Reasoning (ICAPS). Virtual.\\
 \end{tabularx}
 \end{rSection}
 

 \begin{rSection}{Talks at National Conferences}
	\begin{tabularx}{0.95\textwidth} {lp{14cm}}
 \textbf{2021-09-08} & Italian Conference on Computational Logic (CILC). Parma, Italy.\\
 \textbf{2022-10-28} & Doctoral Consortium at the Conference of the Italian Association for Artificial Intelligence (AIxIA). Udine, Italy.\\
 \textbf{2023-11-07} & Italian Workshop on Planning and Scheduling (IPS) at the Conference of the Italian Association for Artificial Intelligence (AIxIA). Rome, Italy.
  \end{tabularx}
 \end{rSection}
 
 \begin{rSection}{Awards and Recognitions}
 \begin{tabularx}{0.95\textwidth} {lp{14cm}}
\textbf{2025} & Honourable Mention for an Outstanding Dissertation Award from the International Conference on Automated Planning and Scheduling 2025.\\
\textbf{2022}  & Honourable Mention of the Leonardo Lesmo Award for the best Italian Master’s thesis in Artificial Intelligence, awarded by the Italian Association for Artificial Intelligence. 
  \end{tabularx}
\end{rSection}

\begin{rSection}{Articles in International Journals}
\begin{enumerate}[leftmargin=5mm]
	\item[J3] \textbf{Optimising Dynamic Traffic Distribution for Urban Networks with Answer Set Programming}. M. Cardellini, C. Dodaro, M. Maratea and M. Vallati - Theory and Practice of Logic Programming,  Volume 24, Issue 4, July 2024, pp. 825-843 - Scimago: Q2 on Artificial Intelligence
	\item[J2] \textbf{Solving Rehabilitation Scheduling Problems via a Two-Phase ASP approach}. M. Cardellini, P. De Nardi, C. Dodaro, G. Galat\`a, A. Giardini, M. Maratea, I. Porro. Theory and Practice of Logic Programming, Volume 24, Issue 2, March 2024, pp. 344-367 - Scimago: Q2 on Artificial Intelligence
	\item[J1] \textbf{Rescheduling Rehabilitation Sessions with Answer Set Programming}. M. Cardellini, C. Dodaro, G. Galat\`a, A. Giardini, M. Maratea, N. Nisopoli and I. Porro. Journal of Logic and Computation, Volume 33, Issue 3, April 2023, pp. 837-863 - Scimago: Q2 on Logic
\end{enumerate}
\end{rSection}
 
\begin{rSection}{Contributions to International Conferences}
\begin{enumerate}[leftmargin=5mm]
	\item[C12] \textbf{Pushing the Envelope in Numeric Pattern Planning}. Matteo Cardellini, and Enrico Giunchiglia. Proceedings of the 22nd International Conference on Principles of Knowledge Representation and Reasoning (KR). 2025 - GGS: A$^{+}$, CORE: A$^{++}$
	
	
	\item[C11] \textbf{Constraint-based In-Station Train Dispatching}. Andreas Schutt, Matteo Cardellini, Jip J. Dekker, Daniel Harabor, Marco Maratea, and Mauro Vallati. Proceedings of the 31st International Conference on Principles and Practice of Constraint Programming (CP). 2025 - GGS: A, CORE: A
	
	
	\item[C10] \textbf{Rolling in Classical Planning with Conditional Effects and Constraints}. M. Cardellini, and E. Giunchiglia. Proceedings of the 34th International Joint Conference on Artificial Intelligence (IJCAI). IJCAI, 2025 - GGS: A$^{++}$, CORE: A$^{++}$
	
	\item[C9] \textbf{Initial Condition Retrieving for Hybrid and Numeric Planning Problems}. M. Cardellini, M. Maratea, F. Percassi and M. Vallati. Proceedings of the 35th International Conference on Automated Planning and Scheduling (ICAPS). AAAI Press, 2025 - GGS: A, CORE: A$^{++}$
	
	
	\item[C8] \textbf{Temporal Numeric Planning with Patterns}. M. Cardellini and E. Giunchiglia. Proceedings of the 39th Annual AAAI Conference on Artificial Intelligence (AAAI). AAAI Press, 2025 - GGS: A$^{++}$, CORE: A$^{++}$
	
	\item[C7] \textbf{Taming Discretised PDDL+ through Multiple Discretisations}. M. Cardellini, M. Maratea, F. Percassi, E. Scala and M. Vallati. Proceedings of the 34th International Conference on Automated Planning and Scheduling (ICAPS). AAAI Press, 2024 - GGS: A, CORE: A$^{++}$
	
	\item[C6] \textbf{Symbolic Numeric Planning With Patterns}. M. Cardellini, E. Giunchiglia and M. Maratea. Proceedings of the 38th Annual AAAI Conference on Artificial Intelligence (AAAI). AAAI Press, 2024 - GGS: A$^{++}$, CORE: A$^{++}$
	
	\item[C5] \textbf{A Framework for Risk-Aware Routing of Connected Autonomous Vehicles via Artificial Intelligence}. M. Cardellini, C. Dodaro, M. Maratea, and M. Vallati. In Proceedings of the 26th IEEE International Conference on Intelligent Transportation Systems (ITSC). IEEE, 2023 - MA: A$^-$
	
	\item[C4] \textbf{A Two-Phase ASP Encoding for Solving Rehabilitation Scheduling}. M. Cardellini, P. De Nardi, C. Dodaro, G. Galat\'a, A. Giardini, M. Maratea and I. Porro. 2021. In Proceedings of the 5th International Joint Conference RuleML+RR. Springer, 2021 - CORE: B
	
	
	\item[C3] \textbf{In-Station Train Movements Prediction: from Shallow to Deep Multi Scale Models}. G. Boleto, L. Oneto, M. Cardellini, M. Maratea, M. Vallati, R. Canepa, D. Anguita. In Proceedings of the 29th European Symposium on Artificial Neural Networks (ESANN). i6doc, 2021 - GGS: B, CORE: B
	
	\item[C2] \textbf{An Efficient Hybrid Planning Framework for In-Station Train Dispatching}. M. Cardellini, M. Maratea, M. Vallati, G. Boleto, and L. Oneto. Proceedings of the 21st International Conference on Computational Science (ICCS). Springer, 2021 - GGS: B, CORE: A
	
	\item[C1] \textbf{In-Station Train Dispatching: A PDDL+ Planning Approach}. M. Cardellini, M. Maratea, M. Vallati, G. Boleto, and L. Oneto. Proceedings of the 31st International Conference on Automated Planning and Scheduling. AAAI Press, 2021 - GGS: A, CORE: A$^{++}$
\end{enumerate}
\end{rSection}

 \begin{rSection}{PhD Thesis}
  \begin{enumerate}[leftmargin=5mm]
	\item[T1] PhD Thesis titled "Symbolic Pattern Planning". Defended on January 9, 2025.	
	\end{enumerate}
 \end{rSection}
 
 
  \begin{rSection}{Member of Program Committees}
	\begin{tabularx}{0.95\textwidth} {lp{14cm}}
 \textbf{2025} & Reviewer. European Conference of Artificial Intelligence \\
 \textbf{2025} & Reviewer. Journal of Applied Logic \\
 \textbf{2023-2025} & Reviewer. International Conference on Automated Planning and Scheduling \\
 \textbf{2023-2026} & Reviewer. International AAAI Conference on Artificial Intelligence 
  \end{tabularx}
 \end{rSection}
 
   \begin{rSection}{Conference Organization}
	\begin{tabularx}{0.95\textwidth} {lp{14cm}}
 \textbf{2025} & Organizer of the "Constraint And Satisfiability-based Planning: an Exploratory Research Workshop" during the International Conference on Planning and Scheduling (CORE: A++, GGS: A) in Melbourne, Victoria, Australia, November 9–14, 2025. \\
  \textbf{2022} & Local Organization of the International Conference on Logic Programming and Non-monotonic Reasoning. Genoa, Italy.
  \end{tabularx}
 \end{rSection}

\break
 \begin{rSection}{Seminars and Invited Talks}
 	\begin{tabularx}{0.95\textwidth} {lp{14cm}}
 \textbf{2025-11-10} & Invited Tutorial on "Planning as SAT: What's New?" during the International Conference on Planning and Scheduling (CORE: A$^{++}$, GGS: A) in Melbourne, Victoria, Australia, November 9–14, 2025.\\
  \textbf{2025-01-25} & University of Genoa. Seminar on "Symbolic Pattern Planning".\\
 \textbf{2024-07-16} & University of Oxford. Seminar on "Symbolic Pattern Planning".\\
 \textbf{2024-02-16} & Bruno Kessler Foundation. Seminar on "Symbolic Pattern Planning".\\
 \textbf{2023-03-01} & University of Huddersfield. Seminar on "An ASP Framework for Efficient Urban Traffic Optimization".\\
 \end{tabularx}

\end{rSection}

\begin{rSection}{Bibliometric Values \tiny{(updated October 14, 2025)}}
\begin{center}
	
\begin{tabular}{ccc}

 & Google Scholar & SCOPUS \\ \hline
\multicolumn{1}{c}{Number of Articles in 5 years} & 20 & 17 \\ \hline
\multicolumn{1}{c}{Number of Citations in 10 years} & 126 & 83 \\ \hline
\multicolumn{1}{c}{H-Index in 10 years} & 
6 & 5 \\ 
\end{tabular}%
\end{center}
\end{rSection}

%
%\pagebreak
%Di seguito le 11 pubblicazioni più la tesi di dottorato che chiedo siano considerate per la valutazione.
%
%\begin{rSection}{Articoli su riviste internazionali}
%\begin{enumerate}[leftmargin=5mm]
%	\item[V1] \textbf{Optimising Dynamic Traffic Distribution for Urban Networks with Answer Set Programming}. M. Cardellini, C. Dodaro, M. Maratea and M. Vallati - Theory and Practice of Logic Programming,  Volume 24, Issue 4, July 2024, pp. 825-843 - Scimago: Q2 on Artificial Intelligence
%	\item[V2] \textbf{Solving Rehabilitation Scheduling Problems via a Two-Phase ASP approach}. M. Cardellini, P. De Nardi, C. Dodaro, G. Galat\`a, A. Giardini, M. Maratea, I. Porro. Theory and Practice of Logic Programming, Volume 24, Issue 2, March 2024, pp. 344-367 - Scimago: Q2 on Artificial Intelligence
%	\item[V3] \textbf{Rescheduling Rehabilitation Sessions with Answer Set Programming}. M. Cardellini, C. Dodaro, G. Galat\`a, A. Giardini, M. Maratea, N. Nisopoli and I. Porro. Journal of Logic and Computation, Volume 33, Issue 3, April 2023, pp. 837-863 - Scimago: Q2 on Logic
%\end{enumerate}
%\end{rSection}
% 
%\begin{rSection}{contributi a convegni internazionali}
%\begin{enumerate}[leftmargin=5mm]
%	\item[V4] \textbf{Rolling in Classical Planning with Conditional Effects and Constraints}. M. Cardellini, and E. Giunchiglia. Proceedings of the 34th International Joint Conference on Artificial Intelligence. IJCAI, 2025 - GGS: A$^{++}$, CORE: A$^{++}$
%
%	\item[V5] \textbf{Initial Condition Retrieving for Hybrid and Numeric Planning Problems}. M. Cardellini, M. Maratea, F. Percassi and M. Vallati. Proceedings of the 35th International Conference on Automated Planning and Scheduling. AAAI Press, 2025 - GGS: A, CORE: A$^{++}$
%	
%	
%	\item[V6] \textbf{Temporal Numeric Planning with Patterns}. M. Cardellini and E. Giunchiglia. Proceedings of the 39th Annual AAAI Conference on Artificial Intelligence. AAAI Press, 2025 - GGS: A$^{++}$, CORE: A$^{++}$
%	
%	\item[V7] \textbf{Taming Discretised PDDL+ through Multiple Discretisations}. M. Cardellini, M. Maratea, F. Percassi, E. Scala and M. Vallati. Proceedings of the 34th International Conference on Automated Planning and Scheduling. AAAI Press, 2024 - GGS: A, CORE: A$^{++}$
%	
%	\item[V8] \textbf{Symbolic Numeric Planning With Patterns}. M. Cardellini, E. Giunchiglia and M. Maratea. Proceedings of the 38th Annual AAAI Conference on Artificial Intelligence. AAAI Press, 2024 - GGS: A$^{++}$, CORE: A$^{++}$
%	
%	\item[V9] \textbf{A Framework for Risk-Aware Routing of Connected Autonomous Vehicles via Artificial Intelligence}. M. Cardellini, C. Dodaro, M. Maratea, and M. Vallati. In Proceedings of the 26th IEEE International Conference on Intelligent Transportation Systems. IEEE, 2023 - MA: A$^-$
%	
%	
%	\item[V10] \textbf{In-Station Train Movements Prediction: from Shallow to Deep Multi Scale Models}. G. Boleto, L. Oneto, M. Cardellini, M. Maratea, M. Vallati, R. Canepa, D. Anguita. In Proceedings of the 29th European Symposium on Artificial Neural Networks. i6doc, 2021 - GGS: B, CORE: B
%	
%	\item[V11] \textbf{In-Station Train Dispatching: A PDDL+ Planning Approach}. M. Cardellini, M. Maratea, M. Vallati, G. Boleto, and L. Oneto. Proceedings of the 31st International Conference on Automated Planning and Scheduling. AAAI Press, 2021 - GGS: A, CORE: A$^{++}$
%\end{enumerate}
%\end{rSection}
%
% \begin{rSection}{tesi di dottorato}
%  \begin{enumerate}[leftmargin=5mm]
%	\item[V12] \textbf{Symbolic Pattern Planning}. M. Cardellini. Discussa il 9 Gennaio 2025.	
%	\end{enumerate}
% \end{rSection}


\end{document}
You