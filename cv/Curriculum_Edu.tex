%%%%%%%%%%%%%%%%%%%%%%%%%%%%%%%%%%%%%%%%%
% Medium Length Professional CV
% LaTeX Template
% Version 2.0 (8/5/13)
%
% This template has been downloaded from:
% http://www.LaTeXTemplates.com
%
% Original author:
% Trey Hunner (http://www.treyhunner.com/)
%
% Important note:
% This template requires the resume.cls file to be in the same directory as the
% .tex file. The resume.cls file provides the resume style used for structuring the
% document.
%
%%%%%%%%%%%%%%%%%%%%%%%%%%%%%%%%%%%%%%%%%

%----------------------------------------------------------------------------------------
%	PACKAGES AND OTHER DOCUMENT CONFIGURATIONS
%----------------------------------------------------------------------------------------

\documentclass{resume} % Use the custom resume.cls style

\usepackage{comment}
\usepackage[left=0.75in,top=0.6in,right=0.75in,bottom=0.6in]{geometry} % Document margins
\usepackage{natbib, bibentry}
\bibliographystyle{plain}
\name{Matteo Cardellini} % Your name
\address{University of Genova, Italy \\ Polytechnic of Torino, Italy} % Your address
\address{matteo.cardellini@polito.it \\ matteo.cardellini@edu.unige.it \\ me@matteocardellini.it } % Your phone number and email


\nobibliography*
\begin{document}
%----------------------------------------------------------------------------------------
%	EDUCATION SECTION
%----------------------------------------------------------------------------------------

\begin{rSection}{Education}

{\bf  Italian National Ph.D. in Artificial Intelligence} \hfill {\em November 2021 - \textit{October 2024}} \\ 
Work Faculty: University of Genova, Italy\\
Administrative Faculty: Polytechnic of Torino, Italy\\
Supervisors: M. Maratea, E. Giunchiglia, and M. Vallati

{\bf University of Genova, Italy} \hfill {\em September 2019 - July 2021} \\ 
Master's Degree on Computer Engineering \\
Curriculum in Artificial Intelligence and Human-Centered Computing \\
Final Grade: 110/110 cum Laude and `Dignit\'a di Stampa" 

{\bf University of Genova, Italy} \hfill {\em September 2016 - September 2019} \\ 
Bachelor's degree on Computer Engineering \\
Final Grade: 110/110 
\end{rSection}


%----------------------------------------------------------------------------------------
%	PUBBLICATIONS AND THESIS
%----------------------------------------------------------------------------------------

\begin{rSection}{Journal Publications}
-- \textbf{Solving Rehabilitation Scheduling Problems via a Two-Phase ASP approach}. M. Cardellini, P. De Nardi, C. Dodaro, G. Galat\`a, A. Giardini, M. Maratea, I. Porro. Theory and Practice of Logic Programming, Volume 24, Issue 2, March 2024, pp. 344-367\\
-- \textbf{Rescheduling Rehabilitation Sessions with Answer Set Programming}. M. Cardellini, C. Dodaro, G. Galat\`a, A. Giardini, M. Maratea, N. Nisopoli and I. Porro. Journal of Logic and Computation, Volume 33, Issue 3, April 2023
\end{rSection}
\begin{rSection}{Conference Publications}
-- \textbf{Taming Discretised PDDL+ through Multiple Discretisations}. M. Cardellini, M. Maratea, F. Percassi, E. Scala and M. Vallati - Proceedings of the 34th International Conference on Automated Planning and Scheduling (ICAPS 2024)\\
-- \textbf{Symbolic Numeric Planning With Patterns}. M. Cardellini, E. Giunchiglia and M. Maratea - Proceedings of the 38th Annual AAAI Conference on Artificial Intelligence (AAAI 2024) \\
-- \textbf{A Framework for Risk-Aware Routing of Connected Autonomous Vehicles via Artificial Intelligence}. M. Cardellini, C. Dodaro, M. Maratea, and M. Vallati. In Proceedings of the 26th IEEE International Conference on Intelligent Transportation Systems (IEEE ITS 2023) \\
-- \textbf{An ASP Framework for Efficient Urban Traffic Optimization}. M. Cardellini. Electronic Proceedings of the 18th Doctoral Consortium on Logic Programming (ICLP DC 2022), 2022 \\
-- \textbf{In-Station Train Dispatching: A PDDL+ Planning Approach}. M. Cardellini, M. Maratea, M. Vallati, G. Boleto, and L. Oneto. Proceedings of the 31st International Conference on Automated Planning and Scheduling. AAAI Press, (ICAPS 2021) \\
-- \textbf{An Efficient Hybrid Planning Framework for In-Station Train Dispatching}. M. Cardellini, M. Maratea, M. Vallati, G. Boleto, and L. Oneto. Proceedings of the 21st International Conference on Computational Science. Springer, (ICCS 2021) \\
-- \textbf{In-Station Train Movements Prediction: from Shallow to Deep Multi Scale Models}. G. Boleto, L. Oneto, M. Cardellini, M. Maratea, M. Vallati, R. Canepa, D. Anguita. In Proceedings of the 29th European Symposium on Artificial Neural Networks. i6doc, 2021\\
-- \textbf{A Two-Phase ASP Encoding for Solving Rehabilitation Scheduling}. M. Cardellini, P. De Nardi, C. Dodaro, G. Galat\'a, A. Giardini, M. Maratea and I. Porro. In Proceedings of RuleML+RR 2021. Springer, 2021\\
-- \textbf{Answer Set Programming in Healthcare: Extended Overview}. M. Alviano, R. Bertolucci, M. Cardellini, C. Dodaro, G. Galat\'a, M. Kamran Khan, M. Maratea, M. Mochi, V. Morozan, I. Porro, and M. Schouten. In Proceedings of IPS-RCRA@AI*IA 2020. CEUR, 2020
\end{rSection}
%--------------------
\begin{rSection}{Theses}
-- \textit{Master's degree}. \textbf{Artificial Intelligence Techniques for Solving the In-Station
Train Dispatching Problem} - \textit{M. Cardellini} - Supervisors: M. Maratea\\
-- \textit{Bachelor's degree}. \textbf{Visual and Data Analytics for the Analysis of Trains' Flux in a Railway Network} - \textit{G. Boleto, M. Cardellini, G. Martino} - Supervisors: L. Oneto, M. Maratea
\end{rSection}
%--------------------
\begin{rSection}{Awards}

-- \textbf{Award AIxIA Leonardo Lesmo 2022}. Special mention for the best Italian Master's Degree Thesis in Artificial Intelligence
\end{rSection}
%--------------------

%--------------------
\begin{rSection}{Scholarships and Projects}
%-- \textbf{IJCAI '24}. \textit{PC Member}. International Joint Conference on Artificial Intelligence \\
-- Post-graduate scholarship on \textbf{Induction and Deduction for Railway Traffic Planning in Small and Medium-sized Stations} - \textit{July 2020 to November 2020 - 5k EUR} - DIBRIS, University of Genova
\end{rSection}
%--------------------

%--------------------
\begin{rSection}{Teaching Activities}
-- \textbf{Artificial Intelligence}. Teaching assistant, Prof. E. Giunchiglia. ING-INF/05, University of Genova (DIBRIS) - 2024\\
-- \textbf{Fondamenti di Informatica}. Teaching assistant. Prof. E. Giunchiglia, ING-INF/05, University of Genova (DIBRIS) - 2023\\
-- \textbf{Intelligent Systems}. Teaching assistant. Prof. M. Maratea, INF/01, University of Calabria (DeMaCS) - 2023\\
-- \textbf{Databases}. Teaching assistant. Prof. A. Boccalatte, Prof. M. Maratea, ING-INF/05, University of Genova (DIBRIS) - 2022\\
-- \textbf{Advanced Artificial Intelligence}. Invited presentations on applications of Planning and Scheduling techniques for real world domains. ING-INF/05, University of Genova (DIBRIS) - 2020, 2021, 2022
\end{rSection}
%--------------------

%--------------------
\begin{rSection}{Invited talks and seminars}
-- \textbf{University of Huddersfield}. Seminar. \textit{An ASP Framework for Efficient Urban Traffic Optimization}. March 1, 2023\\
-- \textbf{Fondazione Bruno Kessler}. Seminar. \textit{Symbolic Pattern Planning}. February 16, 2024
\end{rSection}
%--------------------

%--------------------
\begin{rSection}{Experiences abroad}
-- \textbf{University of Huddersfield}. \textit{Visiting Researcher - From February to Mid May 2023}. Worked in the group of Prof. Mauro Vallati at the Centre for Planning, Autonomy and Representation of Knowledge (PARK)  
\end{rSection}
%--------------------

%--------------------
\begin{rSection}{Supervisor Activities}
-- \textit{October 2022}. \textbf{A. Formica}. Master's Degree in Computer Engineering. \textit{In-Station Train Dispatching via Artificial Intelligence Techniques: Optimisation, Rescheduling and Visualisation}. Co-supervisor with Prof. M. Maratea\\
-- \textit{July 2022}. \textbf{C. Ansaldo, N. Chiesa}. Bachelor's Degree in Computer Engineering. \textit{Artificial Intelligence Techniques for Solving the Shift Scheduling Problem}. Co-supervisor with Prof. M. Maratea
\end{rSection}
%--------------------

\break

%--------------------
\begin{rSection}{Events}
%-- \textbf{IJCAI '24}. \textit{PC Member}. International Joint Conference on Artificial Intelligence \\
-- \textbf{Intelligenza Artificiale '24}. \textit{Guest Reviewer} \\
-- \textbf{ICAPS '23, '24}. \textit{PC Member}. International Conference on Automated Planning and Scheduling \\
-- \textbf{AAAI '23, '24}. \textit{PC Member}. AAAI Conference on Artificial Intelligence \\
-- \textbf{KEPS '23, '24}. \textit{PC Member}. Workshop on Knowledge Engineering for Planning and Scheduling \\
-- \textbf{ECAI '23}. \textit{PC Member}. European Conference on Artificial Intelligence \\
-- \textbf{LPNMR '22}. \textit{Local Organizer}. 18th International Conference on Logic Programming and Non-monotonic Reasoning  
\end{rSection}
%--------------------


%----------------------------------------------------------------------------------------
%	WORK EXPERIENCE SECTION
%----------------------------------------------------------------------------------------
%\begin{rSection}{Social Work}
%
%\begin{rSubsection}{Comunit\'a di Sant'Egidio}{March 2022 - Current}{}{Genoa - IT}
%\item Help to homeless people, bringing them food and medicines.
%\item Help for children and teenagers in studying, doing homework and help against school dropouts.
%\item Voluntary work at nutritional centers and HIV contrast and prevention centers in Malawi, Africa.
%\end{rSubsection}
%
%\end{rSection}
%
%\break

%----------------------------------------------------------------------------------------
%	WORK EXPERIENCE SECTION
%----------------------------------------------------------------------------------------
\begin{rSection}{Work Experience}


\begin{rSubsection}{SurgiQ SRL}{April 2021 - November 2021}{Researcher}{Genova - IT}
\item Built and optimised scheduling systems based on state-of-the-art artificial intelligence's technologies for the scheduling of physiotherapy sessions for the rehabilitation of patients in hospitals.
\end{rSubsection}

\begin{rSubsection}{Secondhand Mobile SRL}{February 2018 - April 2021}{CTO e Co-founder}{Genova - IT}
\item Built a cloud based management system accessed daily by more than 140 customers all-over Italy. 
\item Built and published an iOS/Android App with 65k active downloads. 
\item Helped the company to grow to 14 employees and a yearly revenue of 5 million euros. 
\item Managed 4 software engineers using an Agile methodology continuously delivering improvements to the corporate code-base.
\end{rSubsection}


%------------------------------------------------


\end{rSection}



\nobibliography{publications.bib}

\end{document}
