%%%%%%%%%%%%%%%%%%%%%%%%%%%%%%%%%%%%%%%%%
% Medium Length Professional CV
% LaTeX Template
% Version 2.0 (8/5/13)
%
% This template has been downloaded from:
% http://www.LaTeXTemplates.com
%
% Original author:
% Trey Hunner (http://www.treyhunner.com/)
%
% Important note:
% This template requires the resume.cls file to be in the same directory as the
% .tex file. The resume.cls file provides the resume style used for structuring the
% document.
%
%%%%%%%%%%%%%%%%%%%%%%%%%%%%%%%%%%%%%%%%%

%----------------------------------------------------------------------------------------
%	PACKAGES AND OTHER DOCUMENT CONFIGURATIONS
%----------------------------------------------------------------------------------------

\documentclass{resume} % Use the custom resume.cls style
\usepackage[T1]{fontenc}
\usepackage[utf8]{inputenc}
\usepackage[italian]{babel}
\usepackage{url}
\usepackage{comment}
\usepackage{enumitem}
\setlist{nosep}
\usepackage[left=0.75in,top=0.6in,right=0.75in,bottom=0.6in]{geometry} % Document margins
\usepackage{natbib, bibentry}
\usepackage{tabularx}
\bibliographystyle{plain}
\name{Matteo Cardellini} % Your name
\address{Università degli Studi di Genova, Italia} % Your address
\address{matteo.cardellini@unige.it \\ me@matteocardellini.it } % Your phone number and email


\nobibliography*
\begin{document}


%----------------------------------------------------------------------------------------
%	PUBBLICATIONS AND THESIS
%----------------------------------------------------------------------------------------

\begin{rSectionLower}{A) Organizzazione o partecipazione come relatore a convegni di carattere scientifico in Italia o all'estero}
	\begin{enumerate}
		\item Organizzatore Locale della International Conference on Logic Programming and Non-monotonic Reasoning (CORE: B, GGS: B-). Tenutosi a Genova, Italia dal 6 al 9 Settembre 2022. \\\url{https://sites.google.com/view/lpnmr2022/committees/organizing}
		\item Seminario su invito alla Oxford University su "Symbolic Pattern Planning". 16 Luglio 2024.
		\item Organizzatore del "Constraint And Satisfiability-based Planning: an Exploratory Research Workshop" durante la International Conference of Planning and Scheduling (CORE: A++, GGS: A) a Melbourne, Victoria, Australia dal 9 al 14 Novembre 2025. \\\url{https://icaps25.icaps-conference.org/program/workshops/casp_er/}
		\item Tutorial dal titolo "Planning as SAT: What’s new?" durante la International Conference of Planning and Scheduling (CORE: A$^{++}$, GGS: A) a Melbourne, Victoria, Australia dal 9 al 14 Novembre 2025.
	\end{enumerate}
\end{rSectionLower}

\begin{rSectionLower}{B) Direzione o partecipazione alle attività di un gruppo di ricerca caratterizzato da collaborazioni a livello nazionale o internazionale}
	\begin{enumerate}
		\item 
	\end{enumerate}
\end{rSectionLower}

\begin{rSectionLower}{C) Responsabilità di studi e ricerche scientifiche affidati da qualificate istituzioni pubbliche o private}
	\begin{enumerate}
		\item Coordinatore Scientifico del Laboratorio DiDiLab Congiunto tra l'Università di Genova e ParvaSoft S.p.a. con finalità di ricerca e formazione su intelligenza artificiale applicata alla gestione della logistica integrata.	
	\end{enumerate}
\end{rSectionLower}

\begin{rSectionLower}{D) Responsabilità scientifica per progetti di ricerca internazionali e nazionali, ammessi al finanziamento sulla base di bandi competitivi che prevedano la revisione tra pari}
	\begin{enumerate}
		\item 
	\end{enumerate}
\end{rSectionLower}

\begin{rSectionLower}{E) Direzione o partecipazione a comitati editoriali di riviste, collane editoriali, enciclopedie e trattati di riconosciuto prestigio}
	\begin{enumerate}
		\item 
	\end{enumerate}
\end{rSectionLower}

\begin{rSectionLower}{F) Partecipazione al collegio dei docenti ovvero attribuzione di incarichi di insegnamento, nell'ambito di dottorati di ricerca accreditati dal Ministero}
	\begin{enumerate}
		\item AI-Based Planning. 12h. 3CFU. PhD in Security, Risk and Vulnerability. Università degli Studi di Genova. AA 25/26.
		\item AI-Based Planning. 12h. 3CFU. PhD in Security, Risk and Vulnerability. Università degli Studi di Genova. AA 24/25.
	\end{enumerate}
\end{rSectionLower}

\begin{rSectionLower}{G) Formale attribuzione di incarichi di insegnamento o di ricerca (fellowship) presso qualificati atenei e istituti di ricerca esteri o sovranazionali}
	\begin{enumerate}
		\item 
	\end{enumerate}
\end{rSectionLower}

\break
\begin{rSectionLower}{H) Conseguimento di premi e riconoscimenti per l'attività scientifica, inclusa l’affiliazione ad accademie di riconosciuto prestigio nel settore}
	\begin{enumerate}
		\item Menzione Speciale per la migliore tesi di dottorato conferito dalla International Conference on Automated Planning and Scheduling (ICAPS). 2025.
		\item Menzione Speciale del premio Leonardo Lesmo per la migliore tesi magistrale italiana in Intelligenza Artificiale conferito dalla Associazione Italiana per l'Intelligenza Artificiale. 2022.
	\end{enumerate}
\end{rSectionLower}

\begin{rSectionLower}{I) Risultati ottenuti nel trasferimento tecnologico in termini di partecipazione alla creazione di nuove imprese (spin off), sviluppo, impiego e commercializzazione di brevetti}
	\begin{enumerate}
		\item 
	\end{enumerate}
\end{rSectionLower}

\begin{rSectionLower}{L) Specifiche esperienze professionali caratterizzate da attivita' di ricerca attinenti al settore concorsuale per cui e' presentata la domanda per l'abilitazione
}
	\begin{enumerate}
		\item 
	\end{enumerate}
\end{rSectionLower}

\end{document}
